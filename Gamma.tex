\documentclass[main.tex]{subfiles}
\begin{document}
\chapter{Gamma matrices}
A mastery of gamma matrix manipulation is essential for competence in computing Feynman diagrams for fermions. Let's not worry about where they come from for now. They satisfy a certain number of properties by virtue of making spinors work, being able to work with them is more important than deeply understanding why we use them.

\section{Gamma matrices in varying dimensions}
Let us write $\eta_{\mu \nu}$ for the metric in $d$ dimensions, we will use the mostly minus convention throughout discussion, in the appendix I will provide mostly plus alternatives. 

From on high I decree the characteristic properties of gamma matrices. The \textbf{clifford algebra} in $d$ dimensions is defined as the algebra generated by elements $\gamma^\mu$, $\mu = 0,\cdots,d-1$, satisfying the anticommutation relations
\begin{equation} \label{cliff}
\{\gamma^\mu,\gamma^\nu\} = 2\eta^{\mu \nu}.
\end{equation}
The sign of this defining equation depends on the metric. In the mostly plus metric, eq. \ref{cliff} picks up a minus sign to get the right equations. To be clear: the anticommutation relations are metric independent, the $-$ sign cancels the extra $-$ sign in the metric. We can lower the index via
\[
\gamma_\mu := g_{\mu \nu} \gamma^\nu.
\]
Note, as well, that \ref{cliff} means that
\[
(\gamma^0)^2 = I, \quad (\gamma^i)^2 = -I.
\]
This 
\subsection{Gamma matrix representations in 1,2 and 3 dimensions}
It is both instructive and useful to have some concrete realizations of this algebra in any number of dimensions as matrix algebras. In the $d = 1$ case, $\gamma^0 = I$ does the job, and isn't very interesting. So let's move to the $d = 2$ case. You may know from experience or doing the computation right now that the Pauli sigma matrices work; with 
\[
\sigma^1 = \begin{pmatrix}
0 & 1 \\
1 & 0 
\end{pmatrix}, \quad 
\sigma^2 = \begin{pmatrix}
0 & -i \\
i & 0 
\end{pmatrix}
\]
we define
\begin{equation} \label{g2}
\gamma^0 = \sigma^1, \gamma^1 = -i\sigma^2
\end{equation}
It may be verified directly that these satisfy Eq. \ref{cliff}. What of $\sigma^3$, you ask? From simple intuition about rotations it shouldn't have mattered if we picked any two sigma matrices. So interestingly, in $d=2$, we have at least $3$ matrices from which to construct gamma matrices in $d=2$. I'll come back to this later, but it's interesting to note that we can define the matrix
\[
\Gamma := \gamma^0 \gamma^1
\]
and one easily finds
\[
\Gamma = \sigma^3.
\]
If it didn't matter which of the two Pauli matrices we chose, it is then easily verified that 
\begin{equation}
\{\Gamma,\gamma^\mu\} = 0.
\end{equation}
This matrix also satisfies
\[
\Gamma^2 = 1.
\]
In even dimensions, this matrix will play an important role later. 

Interestingly, if we define the matrix
\[
\gamma^2 = i\Gamma,
\]
then $\gamma^0,\gamma^1,\gamma^2$ satisfy the relation \ref{cliff}. The added $i$ makes sure we have the $-$ sign from the mostly minus metric for $\mu = 2$.

That is, once we moved into matrix representations, we found that the multiplication operation in $1+1$ dimensions gives us an extra $\gamma$ matrix for $2+1$ spacetime dimensions.

\subsection{Gamma matrices representations in arbitrary dimensions}
Now we will be able to provide concrete realizations of the gamma matrices in arbitrary dimensions. We need two tricks. The first is: given an even $d = 2k$, we can construct matrices for $d = 2k+1$, as we just did for $k=1$. The second is similar - given gamma matrices in odd $d=2k+1$, we want to go up to $2k+2$. These techniques leap frog over each other to infinity and we can obtain gamma matrix representations for all $d$.

First, suppose we have $\gamma$ matrices in $d = 2k$ dimensions, $\gamma^\mu$. Define
\begin{equation} \label{g5}
\Gamma = A \prod_{\mu=0}^{2k-1} \gamma^\mu.
\end{equation}
and
\begin{equation}
\gamma^{2k} = i \Gamma.
\end{equation}
We will compute $A$ later, I will show $A = i^{k-1}$ is the right choice to make $\Gamma^2 = 1$. But let's see if  this worked. First, we show that $\Gamma$ anticommutes with everything,. We will do this in cases. For an arbitrary $\gamma^\nu$, observe independently
\begin{equation} \label{ggn}
\Gamma \gamma^\nu = A\gamma^0 \cdots \gamma^\nu \cdots \gamma^{2k-1} \gamma^\nu
\end{equation}
and
\begin{equation} \label{gng}
\gamma^\nu \Gamma = A\gamma^\nu \gamma^0 \cdots \gamma^\nu \cdots \gamma^{2k-1} 
\end{equation}
Ignore the $i$ factor, it does nothing to anticommutativity. 

Now, notice the $\gamma^\nu$ in the middle splits $\Gamma$ in half. It will have an odd number of gamma matrices on one side, and an even on the other. By passing $\gamma^\nu$ all the way from the left and from the right to bring it to its twin $\gamma^\nu$ in the middle, it picks up a minus sign either coming from the left or coming from the right, depending on which side had an odd number of matrices. In total, then, we find
\[
\gamma^\nu \Gamma = - \Gamma \gamma^\nu.
\]
So $\Gamma$ anticommutes with everything. So all we need is $\Gamma^2 = 1$ to seal the deal. By definition, we have
\[
\Gamma^2 = A^2 \gamma^0 \cdots \gamma^{2k-1} \gamma^0 \cdots \gamma^{2k-1}
\]
Now, we want to pair twin $\gamma^\mu$'s in the above expression. $\gamma^0$ has to go through $2k-1$ guys to get to his twin. This gives a $-$ sign. With $\gamma^0$ out of the way, $\gamma^1$ has to go through $2k-2$, which gives a $+$, then $\gamma^2$ moving gives a $-$. So $\gamma^0$ gives a $-$ sign, then we have to move $2(k-1)$ spatial guys through, which give a $-$ sign in pairs. The last spatial guy meets his friend without any work. All in all, this shifting gives a $(-1)(-1)^{k-1}$.
\[
\Gamma^2 = A^2 (-1) (-1)^{k-1}(\gamma^0)^2 \cdots (\gamma^{2k-1})^2
\]
There are odd spatial dimensions, so the square of all the gamma's gives a $-$ sign. So
\[
\Gamma^2 = A^2 (-1)^{k-1}.
\]
Therefore,
\[
A = (\pm i)^{k-1}
\]
makes $\Gamma^2 = 1$. We make the convention of $+i$, so we now have a $\gamma^{2k}$ for $d=2k+1$ dimensions that satisfies \ref{cliff}.

One of the important steps of this proof is noticing that making $\Gamma$ anticommute with everything can't happen (at least in the same way) with an odd number of $\gamma$. When we did Eqs. \ref{ggn} and \ref{gng}, we relied on there being an odd number of $\gamma$ matrices on one side and an even number on the other. For an odd number of gamma matrices, it will only be even-even or odd-odd, and we can't anticommute.

So we've done the $2k\to 2k+1$ step. Now, let's combine representations.

Now we go from the $2k+1 \to 2k+2$ step. You're going to have to forgive my notation here, but in this section, and \textbf{only} this section, I will use capital $\Gamma^\mu$ to represent gamma matrices.

Suppose we have gamma matrices in an odd number of dimensions, $\gamma^i$. Let us now define
\begin{equation} \label{G0i}
\Gamma^0 = -\sigma_1 \otimes I, \quad \Gamma^i = i\sigma_2 \otimes \gamma^i.
\end{equation}
I write $I$ for both the individual identities, ambiguities won't arise because they will be on the left or right. I write $\mb{I} = I\otimes I$. We have to compute $00,0i,ij$ terms. $00$ is easy:
\[
(\Gamma^0)^2 = (-\sigma^1)^2 \otimes I^2 = I \otimes I = \eta^{00} \mb{I}
\]
We expect all $0i$ to anticommute. 
\[
\Gamma^0 \Gamma^i = -i\sigma^1 \sigma^2 \otimes \gamma^i = i\sigma^2 \sigma^1 \otimes \gamma^i = -\Gamma^i \Gamma^0.
\]
Now we want $\{\Gamma^i,\Gamma^j\} = 2\eta^{ij} \mb{I}$.
\begin{align*}
\{\Gamma^i,\Gamma^j\} &= (i\sigma_2)(i\sigma_2) \otimes (\gamma^i \gamma^j) + (i\sigma_2)(i\sigma_2) \otimes (\gamma^j \gamma^i) \\
& = -(\sigma_2)^2 \otimes \{\gamma^i,\gamma^j\} = I \otimes 2\eta^{ij} = 2\eta^{ij} \mb{I}.
\end{align*}
So the $\Gamma^\mu$ satisfy \ref{cliff}, and we have a Clifford algebra in $2k+2$ dimensions. We did it! A comment is in order on this: really I could have tensored together $\gamma^i$ in $d=2$ and $\gamma^{i'}$ in $d = 2k+1$. I chose sign conventions in front of these guys to reproduce the Dirac basis in $d=4$ if we start with \ref{g2} and do \ref{G0i}.

Let's summarize everything we did in this section:
\begin{enumerate} [(1)]
\item We can start with $2\times 2 \gamma$ matrices for $d=2$. This gives us $2\times 2$ matrices for $d=3$. Then $d=4$ becomes $4\times 4$ matrices, $d=5$ is $4\times 4$, then $d=6$ is $8\times 8$ and so on.

\item In even dimensions, we found an extra ``anti-central'' matrix that commutes with everything $\Gamma$, defined via Eq. \ref{g5}. In $d=4$ dimensions, this is usually referred to as $\gamma^5$. It would be $\gamma^4$ but there's history of $x^4$ being time so we skip it. I will sometimes refer to $\Gamma$ in other even dimensions via $\gamma^5$, just because it feels familiar, if there is no risk of ambiguity. I will mainly do it in $d=2$.

\item The method of doing (1) is: with even dimensions, add $i\Gamma$ to the mix. In odd dimensions, tensor in some sigma matrices.
\end{enumerate}

Let's look at some examples. For $d=4$, our prescription gives
\begin{align*}
\gamma^0 
\end{align*}
(fill these in later)

\section{Commutators of Dirac matrices are basically spinor representations of the Lorentz group}
So far we've gone on and on about anticommutators of gamma matrices. Why are we even talking about these matrices in the first place? 

Well, now that we've upgraded these abstract Clifford algebras to matrix algebras, we can define their commutators. What of l their commutators? Do they vanish? Well, in general, no. We'll give them a new name, 
\[
\Sigma^{\mu \nu} := \frac{1}{2}[\gamma^\mu,\gamma^\nu]  
\]
Note that 
%symm anti symm
\begin{equation} \label{sas} 
\gamma^\mu \gamma^\nu = \Sigma^{\mu \nu} + \eta^{\mu \nu}.
\end{equation}
Now let's commute these $\Sigma$ matrices for fun. Using \ref{sas}, since $\eta^{\mu \nu} \propto I$,\footnote{recall we really mean $\eta^{\mu \nu} I$ when we write $\eta^{\mu \nu}$} we have
%commutators of commutators
\begin{equation} \label{comcom}
[\Sigma^{\mu \nu},\Sigma^{\rho \sigma}] = \gamma^\mu \gamma^\nu \gamma^\rho \gamma^\sigma - \gamma^\rho \gamma^\sigma \gamma^\mu \gamma^\nu.
\end{equation}
Now, we recombine these $\gamma$ matrices in a clever way. If we recombined $\mu \nu$, we'd just end up with $[\Sigma^{\mu \nu}, \Sigma^{\rho \sigma}]$ again. I'm going to guess that these $\Sigma^{\mu \nu}$ are a Lie algebra, so what I want to find is terms linear in $\Sigma$. Any term linear in $\Sigma$ would vanish with the naive $\mu \nu$ and $\rho \sigma$ combination.

So what can we do?  Well, I'm not intuitive with algebra, so let me explain how I did it. First I didn't have the form \ref{comcom} at all. After thinking about commutators and anticommutators though, it makes the most sense. 

Once we have \ref{comcom}, it seems productive to try to get the terms in \ref{comcom} to cancel, by moving gamma matrices on the right term through each other, and picking up anti-commutators.

On reflection, though, it's the obvious thing to do. If we want terms linear in $\Sigma$, we need terms bilinear in $\gamma$. We can get that by anti-commuting the $\gamma$ matrices to find a  $g^{\mu \nu}$, and by making the terms quartic in $\gamma$ vanish in \ref{comcom} we can achieve our goal. 

So we move gamma matrices around in \ref{comcom}. All in all, we need to make $4$ switches, so there is an overall $(-1)^4$ sign that will make sure the original contributions really ``cancel''. So \ref{comcom} becomes
\[
=-2( \eta^{\sigma \mu} \gamma^{\rho} \gamma^{\nu} -\eta^{\rho \mu} \gamma^\sigma \gamma^\nu + \eta^{\sigma \nu} \gamma^\mu \gamma^\rho - \eta^{\rho \nu} \gamma^\mu \gamma^\sigma).
\]
I recommend that you do this yourself. Notice that we don't get any original pairings of indices, because we never swapped any matrices with their cousins\footnote{Obviously, e.g. $\mu$ and $\nu$ are cousins but not $\mu \rho$.}

Now, we replace every $\gamma^a \gamma^b$ using its commutator and anticommutator, via \ref{sas}. It is then apparent that all $g^2$ terms cancel, check this for yourself.\footnote{I don't think it is not enough to argue that they are all symmetric in $\mu \nu$, while the original $\Sigma$ commutator is antisymmetric. One can take antisymmetric combinations of symmetric objects, e.g. $a^{\mu \nu} := \eta^{\mu \nu} - \eta^{\nu \mu}$. I can't say the $g$ contributions to $a$ cancel because $a$ is antisymmetric, or I would have $a = 0$.} So we basically can just replace $\gamma^\mu \gamma^\nu \to \Sigma^{\mu \nu}$. We then have
\[
[\Sigma^{\mu \nu},\Sigma^{\rho \sigma}] = -2 \left(
\eta^{\sigma \mu} \Sigma^{\rho \nu} -\eta^{\rho \mu} \Sigma^{\sigma \nu} + \eta^{\sigma \nu} \Sigma^{\mu \rho} - \eta^{\rho \nu} \Sigma^{\mu \sigma}
\right)
\]
This should look familiar!! In fact, if we define
%srep
\begin{equation} \label{srep}
\sigma^{\mu \nu} := \frac{i}{2} \Sigma^{\mu \nu} = \frac{i}{4} [\gamma^\mu,\gamma^\nu]
\end{equation}
then the $\sigma^{\mu \nu}$ furnish a representation of the Lorentz group!\footnote{lots of minus signs and $i$'s show up but when the smoke clears it works out} The representation $\sigma^{\mu \nu}$ in \ref{srep} is often referred to as the spinor representation of the Lorentz group. 

\section{Traces of gamma matrices in even dimensions}
We will come to speak of gamma matrices in even dimensions soon. In this section, we specialize to even dimensions because of the anticommutant $\gamma^5$, which satisfies
\[
\{\gamma^5,\gamma^\mu\} = 0, \quad (\gamma^5)^2 = 1.
\]
In $d$ dimensions, we have
\[
\tr I = d.
\]
By symmetry of the trace,
\[
\tr \gamma^{\mu} \gamma^\nu = \tr \gamma^\mu \gamma^\nu,
\]
so
\[
\tr \{\gamma^\mu,\gamma^\nu\} = d g^{\mu \nu}, \quad \tr \Sigma^{\mu \nu} = 0.
\]
This works in odd $d$ as well so far. But let's start using $\gamma^5$, which we've only seen in even dimensions\footnote{We will have more to say about this later}. 

Take an arbitrary number of $\gamma$ matrices, $\gamma^{\mu_1},\cdots,\gamma^{\mu_n}$. Let's take the trace of their product. Since $(\gamma^5)^2 = 1$, we have
\[
\tr \, \gamma^{\mu_{1}} \cdots \gamma^{\mu_n} = \tr \, (\gamma^5)^2 \gamma^{\mu_1} \cdots \gamma^{\mu_n}.
\]
Now, we pass one of these $\gamma^5$'s to the right side with cyclicity of trace, pac-man style. The other $\gamma^5$ goes through $n$ gamma matrices. They recombine and annihilate, squaring to $1$ again. The gamma matrix that worked hard picks up a $(-1)^n$, so we find
\[
\tr \gamma^{\mu_1} \cdots \gamma^{\mu_n} = (-1)^n \tr \gamma^{\mu_1} \cdots \gamma^{\mu_n}
\]
so it vanishes if $n$ is odd.

Similarly, if we start with $\gamma^5 \gamma^{\mu_1} \cdots \gamma^{\mu_n}$, the same trick reveals its trace vanishes if $n$ is odd. In summary,
\begin{enumerate}
\item $\tr \gamma^5$ (odd \# of $\gamma$ matrices) $ = 0$,

\item $\tr$ (odd \# of $\gamma$ matrices) $= 0$.
\end{enumerate}
This will vastly simplify certain computations. Roughly speaking, half of the traces we could ever compute need to compute vanish.
\end{document}