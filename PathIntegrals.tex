\documentclass[main.tex]{subfiles}
\begin{document}
\chapter{Path Integrals}
There are two major approaches to defining quantum field theory. Historically, the first is via Dyson's equation and the S-matrix. These will be good things to know, so we will cover them in the next chapter. First, we will go over what has become the standard way to think about field theories: the path integral.

As will become clear in the next chapter, these theories have one thing in common: they make the same predictions. If you don't like path integrals, you can think of the recipe developed in this chapter as an easy way to generate correlation functions. We will see in the LSZ chapter that these help us define scattering amplitudes, which in turn give us the things we care about computing: cross sections, decay rates, and general $n$-particle processes.

It's good to have a review of the path integral for quantum mechanics before we do it for quantum fields. The main difference here is that the degrees of freedom go from finitely many (position indices, $i=1,2,3$) to infinitely/continually many (position itself $x$).
\section{The path integral in quantum mechanics}
There are a million resources on path integration in quantum mechanics, you don't need to read mine. But I will use language that I hope makes the generalization to field theory clear.


\end{document}