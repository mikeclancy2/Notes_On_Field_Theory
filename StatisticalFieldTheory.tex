\documentclass{book}
\usepackage[utf8]{inputenc}
\usepackage[utf8]{inputenc}
\usepackage{geometry}
 \geometry{
 a4paper,
 total={170mm,247mm},
 left=20mm,
 top=20mm,
 }
\usepackage[utf8]{inputenc}
\usepackage{braket}
\usepackage{amsmath}
\usepackage{amssymb}
\usepackage{amsthm}
\usepackage{mathtools}
\usepackage{amsfonts}
\usepackage{physics}
\usepackage{hyperref}
\usepackage{tikz-cd}
\usepackage{enumerate}
\usepackage{tikz-feynman}
\tikzfeynmanset{compat=1.0.0}
\usepackage{graphicx}
\usepackage{slashed}
\usepackage{xfrac}
%\usepackage{xcolor}
%\pagecolor[rgb]{0,0,0}
%\color[rgb]{0,0.7,0.7}
\newcommand{\cm}{\mathbb{C}}
\newcommand{\re}{\mathbb{R}}
\newcommand{\mb}{\mathbf}
\newcommand{\proj}[1]{\ensuremath{{\ket{#1}\bra{#1}}}}
\newtheorem{defn}{Definition}
\newtheorem{thm}{Theorem}
\newtheorem{prop}{Proposition}
\newtheorem{rmk}{Remark}
\newtheorem{conj}{Conjecture}
\newtheorem{lemma}{Lemma}
\newtheorem{cor}{Corollary}
\newenvironment{claim}[1]{\par\noindent\textit{Claim.}\space#1}{}
\newenvironment{claimproof}[1]{\par\noindent\textit{Subproof.}\space#1}{\hfill $\square$}
\renewcommand{\qedsymbol}{\ensuremath{\blacksquare}}

\begin{document}

\chapter{Statistical Field Theory}
test test
There is a striking similarity between the form of the partition function in statistical field theory and the action in quantum field theory. The partition function in statistical field theory is given by
\[
Z[\beta] = \int D \phi \exp{- \beta \int d^d x \, \mathcal{F} (\phi)}
\]
After Wick rotating our quantum field theory path integral, we obtain
\[
Z[J] = \int D \phi \exp{-  \int d^4 x \, \mathcal{L} (\phi,J)}
\]
I'm not quite ready to write about this yet. First let's do some problems. Here is an exercise from Peskin {\&} Schroeder (9.2a):

\textbf{Problem 9.2:} \newline
(a) Evaluate the quantum statistical partition function
\[
Z = \tr [e^{-\beta H}]
\]
using the strategy of Section 9.1 (p{\&}s) for evaluating the matrix elements of $e^{-iHt}$ in terms of functional integrals. Show that one again finds a functional integral, over functions defined on a domain that is of length $\beta$ and periodically connected in the time direction. Note that the Euclidean form of the Lagrangian appears in the weight. 

\textbf{Solution:} \newline
We start in the position basis (since we're doing a method similar to 9.1). The trace is 
\[
\tr e^{-\beta H} = \int dx \bra{x} e^{-\beta H} \ket{x}
\]
Now we divide up $\beta$ into a million pieces:
\[
e^{-\beta H} = e^{-H\epsilon} \cdots e^{-H\epsilon}
\]
where $\epsilon = \beta/N$ (large N).
We insert a resolution of the identity in the position basis between each factor:
\[
\tr e^{-\beta H} = \int \prod_{j=1}^{N-2} dq_j \bra{q_j} e^{-\epsilon H} \ket{q_{j+1}}
\]
with the last term being $q_1 = q_N$ (the notation gets sloppy here but it's fine). In order to evaluate this we insert momentum eigenstates; the matrix elements of $\bra{x} e^{-\beta H} \ket{y}$ are 
\[
\bra{x} e^{-\beta H} \ket{y} = \frac{1}{2\pi \hbar} \sqrt{\frac{2\pi m}{\epsilon}} e^{-m(x-y)^2/2\epsilon \hbar^2 -\epsilon V(y)},
\]
so the trace becomes\footnote{I am following Hitoshi on this step - where did the $m$ go?}
\[
Z = \int \prod_j dx_j \frac{1}{\sqrt{2\pi \hbar^2 \epsilon}} \exp{-S_E\hbar}
\]
where we have defined
\[
S_E = \hbar \sum_{j=1}^N \frac{m(x_i - x_{i-1})^2}{2\epsilon \hbar^2} + \epsilon V(x_i).
\]
Putting $\beta$ back in, this is
\[
S_E = \hbar \sum_{j=1}^N \frac{m(x_i - x_{i-1})^2}{2(\hbar \beta/N)} + V(x_i)
\]
Now comes the physical insight. We're doing an integral over all paths. If we assume that these $x_i$ aren't arbitrary but in the limit $N\to \infty$ look like differentiable paths, we can turn this difference into a derivative. We define the parameter $\tau$ to run from $0$ to $\hbar$, we have
\[
S_E = \int_0^{\hbar \beta} \left[\frac{m}{2} \left( \frac{dx}{dt} \right)^2 + V(x) \right] d\tau
\]
where we integrate with respect to the constraint $x(\hbar \beta) = x(0)$ to make sure we are taking the trace. So we may write
\[
Z = \int \prod_j dx_j \frac{1}{C(\beta)} \exp{-S_E}
\]
where $C(\beta)$ is some nasty divergent function.

\textbf{b} Evaluate this integral for a simple harmonic oscillator, 
\[
L_E = \frac{1}{2} \dot{x}^2 + \frac{1}{2} \omega^2 x^2 ,
\]
by introducing a Fourier decomposition of $x(t)$:
... 

\textbf{c} Generalize this construction to field theory. Show that the quantum statistical partition function for a free scalar field can be written in terms of a functional integral. The value of this integral is given formally by the usual inverse root determinant of the Klein-Gordon operator. \newline
\textbf{Solution:} \newline
The same construction, instead of integrating over position eigenstates, but field configurations, produces the expression
\[
Z = \int D\phi \frac{1}{C(\beta)} \exp(-S_E),
\]
so the usual gaussian integral trick does the same thing. However, in this case, the Euclidean action is given by 
\[
\int_0^\beta \int d^d x \; \mathcal{L} \; d\tau
\]

Note to self: finish this problem. Extremely interesting.

\end{document}