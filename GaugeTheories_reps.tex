\documentclass{book}
\usepackage[utf8]{inputenc}
\usepackage[utf8]{inputenc}
\usepackage{geometry}
 \geometry{
 a4paper,
 total={170mm,247mm},
 left=20mm,
 top=20mm,
 }
\usepackage[utf8]{inputenc}
\usepackage{braket}
\usepackage{amsmath}
\usepackage{amssymb}
\usepackage{amsthm}
\usepackage{mathtools}
\usepackage{amsfonts}
\usepackage{physics}
\usepackage{hyperref}
\usepackage{tikz-cd}
\usepackage{enumerate}
\usepackage{tikz-feynman}
\tikzfeynmanset{compat=1.0.0}
\usepackage{graphicx}
\usepackage{slashed}
\usepackage{xfrac}
%\usepackage{xcolor}
%\pagecolor[rgb]{0,0,0}
%\color[rgb]{0,0.7,0.7}
\newcommand{\cm}{\mathbb{C}}
\newcommand{\re}{\mathbb{R}}
\newcommand{\mb}{\mathbf}
\newcommand{\proj}[1]{\ensuremath{{\ket{#1}\bra{#1}}}}
\newtheorem{defn}{Definition}
\newtheorem{prop}{Proposition}
\newtheorem{rmk}{Remark}
\newtheorem{conj}{Conjecture}
\newtheorem{lemma}{Lemma}
\newtheorem{cor}{Corollary}
\newenvironment{claim}[1]{\par\noindent\textit{Claim.}\space#1}{}
\newenvironment{claimproof}[1]{\par\noindent\textit{Subproof.}\space#1}{\hfill $\square$}
\renewcommand{\qedsymbol}{\ensuremath{\blacksquare}}

\begin{document}
\chapter{Gauge Theories: Representation Matters}
In the previous chapter, we started with a field $\phi$, an introduced $N$ local degrees of freedom which we did gauge transformations $SU(N)$ to at each point, and the theory of gauge fields naturally followed.

However, more generally, we could ask what happens if we loosen this somewhat: introduce local degrees of freedom, and have the group $G$ act on them. So let's make a general (conversational) definition.

\begin{defn}
Given some field theory $\phi(x)$, and group $G$ we \textbf{gauge a symmetry} $G$ or \textbf{add a gauge symmetry} $G$ by introducing local degrees of freedom to $\phi$ which transform according to a representation of $G$, and do everything we did in the last section to account for this.
\end{defn}

Here's what this means. If we started  with some matrix Lie group $G$ of $N \times N$ matrices, we don't need to just consider introducting $N$ degrees of freedom. We could consider some $M \times M$ representation $R$ of $G$, call its elements $U_R$, and introduce $M$ local degrees of freedom to $\phi$. Those degrees of freedom then transform according to 
\[
\phi_i (x)\to U_R (x)_{ij} \phi_i (x)
\]
Instead of introducing a $\mathfrak{g}$-valued connection $A_\mu$, instead we introduced a $\mathfrak{g}$-representation valued connection $A_\mu^R$. The entire theory is then constructed basically exactly as it was in the previous section. We will cover the details of doing this later in this chapter. First, we will review representations and introduce some defnitions what will be useful\footnote{I am following roughly the presentation of Srednicki\cite{Sred1}}.

For a proper treatment of representation theory, see Hall\cite{Hall1}. I will be pretty loose with how I define things, since the introduction of technical and more precise language doesn't help too much.

\newpage
\section{Lie group and Lie algebra representations}
In short, we will circumvent Lie group representation theory in total - we will only do Lie algebra representations and then exponentiate those to get Lie group representations. A \textbf{Lie algebra representation} $\pi_R: \mathfrak{g} \to R$ is a Lie algebra homomorphism from a Lie algebra $\mathfrak{g}$ to a collection on $D(R) \times D(R)$ traceless Hermitian matrices which is a Lie algebra homomorphism. The number $D(R)$ is called the dimension of the representation. Note that the dimension of the collection of matrices as a vector space is different from the dimension of the vector space upon which they act. 

In order to be a homomorphism, we must preserve commutators and be linear. Let $T^a$ be a basis for the original Lie algebra. Write $T^a_R = \pi_R (T^a)$, and let $f^{abc}$ denote the structure constants of the original algebra. Then we must have
\begin{equation} \label{lalgrep}
[T^a_R,T^b_R] = i f^{abc} T^c_R
\end{equation}
in order for $\pi_R$ to be a representation. In fact, sometimes we aren't so worried about constructing maps explicitly, we will just start with some $T^a_R$ and see that they form a representation in the sense of (\ref{lalgrep}). In either case, we also refer to $T^a_R$ as the representation, because by linear extension they specify a map $\pi_R$. \footnote{Srednicki says they must be traceless and Hermitian. Follow up on this.} Let's knock some definitions out of the way.
\begin{itemize}
\item A representation $T^a_R$ is called \textbf{real} if 
\begin{equation} \label{realrep}
- (T^a_R)^* = T^a_R,
\end{equation}
or if there is a unitary transformation $U$ such that $T^a_R \to U^{-1} T^a_R U =: \tilde{T}^a_R$ that makes the $\tilde{T}^a_R$ satisfy (\ref{realrep}).
\item A representation $T^a_R$ is called \textbf{pseudoreal} if there is a unitary transformation $V$ such that
\[
-(T^a_R)^* = V^{-1} T^a_R V
\]
for all the $a$'s at once.
\item A representation $T^a_R$ is called \textbf{complex} if it is not real or pseudoreal. In this case, the \textbf{complex conjugate representation} $\overline{R}$ is defined by
\[
T^a_{\overline{R}} = - \left( T^a_R \right)^*
\]
\end{itemize}
To see why we use these terms, let's look at (\ref{lalgrep}). If $T^a_R$ is real, then the Lie algebra generated by $(T^a_R)^*$ is the same Lie algebra as the original one (loosely $\pi = \pi^*$). If $T^a$ is pseudoreal, it is unitarily equivalent to its conjugate (loosely $\pi \equiv \pi^*$). Otherwise we have something like $\pi \neq \pi^*$. Complex representations sound rather pathological. Quite the contrary - some of the simplest representations are complex. We will demonstrate one now.

In the cases we consider, we will have that $G$ is a matrix Lie group, so $\mathfrak{g}$ is a matrix lie algebra, e.g. $\mathfrak{g} = \mathfrak{su}(N)$. In the matrix case, we call $\mathfrak{g}$, or its basis $T^a$, the \textbf{defining} or \textbf{fundamental} representation. This representation is therefore $N$-dimensional. blah blah $SU(2)$ pseudoreal, $SU(\geq 3)$ complex. 
\end{document}