\documentclass[main.tex]{subfiles}
\begin{document}
\chapter{Anomalies}
An anomaly is a quantum mechanical violation of Noether's theorem. Heuristically, if we start with a current  $j^\mu$ satisfying
\[
\partial_\mu j^\mu = 0
\]
classically, an anomaly is a term that shows up in the quantum theory,
\[
\partial_\mu j^\mu = A \neq 0.
\]
To this end, let us consider the axial transformation, which has infinitesimal form
\[
\delta \Psi = i \gamma_5 \Psi.
\]
We define the axial current $j^{\mu 5}$ via (fix this later)
\begin{equation} \label{axialcurrent}
j^{\mu 5}= - \frac{\delta \mathcal{L}}{\delta (\partial_\mu \Psi)} \delta \Psi.
\end{equation}
Then classically 
\[
\partial_\mu j^{\mu 5} = 0.
\]
Quantum mechanically, we will find the story is not so simple. 

We saw that the the W-T identities required the measure to be invariant under the transformation $\phi_a \to \phi_a + \delta \phi_a$. We will find that this is violated by the axial transformation. So we have to choose a path integral measure that is also symmetric under axial transformations.

\section{The Chiral Anomaly}
This is historically the first anomaly that appeared in calculations (and in experiments). (refneeded) find some sources.

Consider the typical measure
\begin{equation} \label{pimeas}
D\psi D \overline{\psi}.
\end{equation}
Under the transformation
\begin{equation} \label{axialtrans}
\psi \to \psi' = \psi + i \epsilon \gamma^5 \psi,
\end{equation}
and the same for $\overline{\psi}$, we will pick up a Jacobian.

To do this, it's actually easier to rewrite the measure less abstractly and more concretely. Instead of integrating over field configurations, we will expand field configurations in a basis and integrate over the values of those coefficients. 

So consider the operator $\slashed{D}$ for the spinor. This operator has eigenstates, which satisfy
\[
i \slashed{D} \phi_n = \lambda_n \phi_n.
\]
These are complex numbers - the grassmann variables appear in the coefficients. An arbitrary spinor may be expanded
\[
\psi (x) = \sum_n a_n \phi_n (x),
\]
for anticommuting $a_n$. We expand $\overline{\psi}$ in the same way:
\[
\overline{\psi} (x) = \sum_n \overline{b}_n \overline{\phi}_n (x).
\]
These eigenspinors can be chosen to be orthogonal by diagonalizing any degenerate subspaces,
\[
\int d^d x \overline{\phi}_n \phi_m = \delta_{nm}
\]
Let's consider a massless action
\[
S = \int d^d x i \overline{\psi} \slashed{D} \psi.
\]
In the basis expansion we chose this becomes
\[
S = \sum_n \lambda_n \overline{b}_n a_n.
\]
Therefore, the path integral measure (\ref{pimeas}) becomes
\[
\int D \psi D \overline{\psi} = \prod_n \int d\overline{b}_n da_n.
\]
The Euclidean path integral then becomes
\[
\int D \overline{\psi} D \psi = \prod_n \int d\overline{b}_n da_n e^{-sum_m \lambda_m \overline{b}_m a_m}.
\]
But Grassmann integration is easy, we know that this just becomes
\[
\prod_n \delta_{mn} \delta_{mn} \lambda_m \overline{b}_m a_m = \prod_n \lambda_n = \det i \slashed{D}.
\]
The $-$ sign in the exponential went away because we had to pass the $\overline{b}$ through the $a$, really this path integral can be equivalently thought of as a bunch of derivatives since derivatives and integrals are the same thing with Grassmann variables.

Now, let's perform an infinitesimal position dependent axial rotation on everything.
\[
\delta \psi (x) = i \epsilon (x) \gamma^5 \psi (x).
\]
This can be expanded using the eigenfunctions as usual:
\[
\delta \psi (x) = i \epsilon(x) \sum_m a_m \gamma^5 \phi_m.
\]
If we also try to write $\delta \phi$ by itself, we would have parameters $\delta a_n$:
\[
\delta \phi (x) = \sum_n \delta a_n \phi_n
\]
Exploiting orthogonality relations we find therefore that 
\[
\delta a_n = \left( 
i \int d^4 x \epsilon (x) \overline{\phi}_n \gamma^5 \phi_m
\right) a_m
\]
Let's write $X_{nm}$ for that big boy in parentheses.

We therefore have, in total,
\[
\psi' = \sum_n a_n' \phi_n = \sum_n \left( a_n + \delta a_n \right)\phi_n = \sum_n \left( a_n + X_{nm} a_m \right) = \sum_n \left( \delta_{mn} + X_{nm} \right) a_n \phi_n
\]
Expanding all math in the $\phi_n$ basis we observe that we may write
\[
\psi' = (1 + X) \psi.
\]
Under a transformation $\psi \to \psi'$, we must insert an inverse determinant, this should be reasonable from experience with fermionic integrals, and I should do this later, but the important part is that we obtain
\[
\prod_n \int d\overline{b}_n da_n = \prod_n  \int d\overline{b}'_n da'_n \frac{1}{(\det (1 + X))^2}.
\]
I'm just following Tong here, but it's interesting to note at this point that the fact that $\psi$ and $\overline{\psi}$ transform with the same sign in front of the $i$ for axial transformations means that we get this squared determinant, normally the fact that they are conjugates of one another would make these cancel each other out.

Using
\[
\frac{1}{\det A} = e^{-\tr \ln A}, 
\]
since $A = 1+X$ we have $\ln (1 + X) = X$, we have
\[
\frac{1}{\det A} = e^{-\tr X}.
\]
Let's call the above $J$. Taking the trace of $X$ is easy, since we know its matrix elements. It becomes
\begin{equation} \label{detJ}
J = \exp{ 
-i \int d^d x \epsilon (x) \sum_n \overline{\phi}_n (x) \gamma^5 \phi_n (x)
}
\end{equation}
So far we've made no attempt to regularize our theory. This is the first place we do so. The exponent of \ref{detJ} is perhaps not well defined (Fujikawa). Inside of $J$ we insert a regularization scale $\Lambda$. write \ref{detJ} as $J = e^{iY}$. Define
\[
Y(\Lambda) = \int d^d x \epsilon \sum_n \overline{\phi}_n (x) \gamma^5 \phi_n (x) e^{-\lambda^2_n/\Lambda^2}.
\]
Cleverly, then, as $\Lambda \to \infty$, $Y(\Lambda) \to Y$. But independently of this, we may replace the eigenvalues with $\slashed{D}$. $Y$ then becomes
\begin{equation} \label{Yexp}
Y = \lim_{\Lambda \to \infty} \int d^d x \epsilon (x) \sum_n \overline{\phi}_n (x) \gamma^5 e^{-(i\slashed{D})^2/\Lambda^2} \phi_n.
\end{equation}
So far, we've been using a basis of functions $\phi_n$ which were eigenfunctions of $\slashed{D}$. We have found this convention useful for a number of reasons. However, we will instead switch to the plane wave basis instead of $\phi_n$. This means $\sum_n \to \int \frac{d^d k}{(2\pi)^d}$ and $\phi_n (x) \to e^{ikx}$. Then (\ref{Yexp}) becomes:
\[
Y(\Lambda) = \int d^d x \epsilon (x) \int \frac{d^d k}{(2\pi)^d} \tr \left(
\gamma^5 e^{-ikx} e^{\slashed{D}^2/\Lambda^2} e^{ikx} 
\right).
\]
Now, let's spend some time with this $e^{\slashed{D}^2/\Lambda^2}$. The exponent can be written
\[
\slashed{D}^2 = \gamma^\mu \gamma^\nu D_\mu D_\nu.
\]
Products of matrices can be replaced by a commutator plus and anticommutator; we have then
\begin{equation} \label{sd2}
\slashed{D}^2 = \frac{1}{2} \{ \gamma^{\mu},\gamma^\nu \} D_\mu D_\nu + \frac{1}{2} [\gamma^\mu,\gamma^\nu] D_\mu D_\nu.
\end{equation}
Now, pay attention carefully: the anticommutator gives us $2\eta^{\mu \nu}$. So the first term is $D^2$. Now, since the $\mu,\nu$ are summed over, we can switch the commutator from the gamma matrices over to the $D_\mu$ by swapping indices on one of the terms (try for yourself). But we have another name for this commutator;
\[
-ig F_{\mu \nu} := [D_\mu,D_\nu].
\]
Therefore (\ref{sd2}) becomes
\begin{equation} \label{d2}
\slashed{D^2} = D^2 - \frac{ig}{2} \gamma^\mu \gamma^\nu F_{\mu \nu}.
\end{equation}
We can make some further simplifications. First, note
\begin{equation} \label{dk}
e^{-ikx} D_\mu e^{ikx} = i k_\mu + ig A_\mu.
\end{equation}
The quantity we wish to simplify is now
\begin{equation} \label{man}
\gamma^5 e^{-ikx} e^{\slashed{D}^2} e^{ikx}.
\end{equation}
Using both \ref{d2} and \ref{dk}, we find that expanding \ref{man} to all orders requires the BCH formula. It will turn out that they vanish in the $\Lambda \to \infty$ limit, as per Tong. I will include more about this later. So let's just drop them and further manipulations to omit terms which becomes unity in the $\Lambda \to \infty$ limit.

It is natural to group the gamma matrices together for \ref{man}. We then have
\[
\gamma^5 e^{-ikx} e^{\slashed{D}^2/\Lambda^2} e^{ikx} = \gamma^5 e^{(i(k_\mu + g A_\mu(x)^2} e^{\frac{-ig}{2} \gamma^\mu \gamma^\nu F_{\mu \nu}/\Lambda^2.}
\]
I'm a little uncertain about this next step but for now will just go with it. Let's drop the $g A_\mu$ in the first term. Then the above becomes
\[
= \gamma^5 e^{-k^2/\Lambda^2} \left(
\gamma^\mu \gamma^\mu F_{\mu \nu} \frac{1}{\Lambda^2} + \cdots
\right)
\]
where we have expanded the exponential in $F$. Higher order terms with $\Lambda$ will vanish in the $\Lambda \to \infty$ limit; we will be working in $d = 2$. Recalling where we came from, we therefore have
\begin{equation} \label{phigamphi}
\sum_n \overline{\phi}_n \gamma^5 \phi_n = \lim_{\Lambda \to \infty} \int \frac{d^d k}{(2\pi)^d} e^{-k^2/\Lambda^2} \tr (\frac{-ig}{2}\gamma^5 \gamma^\mu \gamma^\nu F_{\mu \nu} \frac{1}{\Lambda^2})
\end{equation}
Let us choose the gamma matrix convention so that
\[
\tr \gamma^5 \gamma^\mu \gamma^\nu = 2 i\epsilon^{\mu \nu}.
\]
I have to be more careful here...

In any case, let's go back to \ref{phigamphi}. This is a gaussian integral in $k$. So we pick up a $\pi$ from the $2d$ integral in the numerator, and a $\Lambda^2$ in the numerator as well. Therefore \ref{phigamphi} is given by
\[
= \frac{1}{(2\pi)^2} g \pi 2 \frac{1}{2} \epsilon^{\mu \nu} F_{\mu \nu} = \frac{e}{4\pi} \epsilon^{\mu \nu} F_{\mu \nu}
\]
identifying $g = e$. We will see in a second that this is exactly the anomaly. Recall
\[
Y = \int d^2 x \epsilon(x) \frac{e}{4\pi} \epsilon^{\mu \nu} F_{\mu \nu}
\]
Lastly, put this in $J = e^{iY}$. Then under the axial transformation \ref{axialtrans}, we pick up a $J^2$, meaning the action acquires a constant term
\[
\mathcal{L}_{anom} = \frac{\epsilon}{2\pi} \epsilon^{\mu \nu}F_{\mu \nu}.
\]
Since $F$ and $\epsilon$ are both antisymmetric, this can also be written
\[
\delta \mathcal{L}_{anom} = \frac{\epsilon}{\pi} F^{01} = \frac{\epsilon}{\pi} E(x)
\]
Now, we see exactly how this violates axial symmetry. Under a chiral transformation, we have
\[
\partial_\mu j^{\mu 5} = \delta \mathcal{L} (x)
\]
with $j^\mu$ the axial current given by \ref{axialcurrent}. If we look at $\delta \mathcal{L}$ just classically, all of its variations can only come from $\mathcal{L}$ itself. In the quantum case, we have also the path integral measure which might vary, and in this case, instead of $\delta mathcal{L} = 0$, we have an anomalous shift in the Lagrangian. We therefore have
\begin{equation} \label{11chiralanomaly}
\partial_\mu j^{\mu 5} = \frac{e}{2\pi} \epsilon^{\mu \nu} F_{\mu \nu}.
\end{equation} 
This is the chiral anomaly in $1+1$ dimensions. Let us review what has happened.
\subsection{Anomalous W-T diagram in 1+1 and 3+1 dimensions}
In this section, I observe the properties of the vacuum polarization diagram in $1+1$ dimensions for a $U(1)$ gauge theory.

That is, consider the diagram

(figure out how to do feynman diagrams, $\gamma^\mu$ at one vertex and $\gamma^5 \gamma^\nu$ at the other, check PnS page 654 (refneeded) for details. 

This is a diagram that arises as the result of a background gauge field. We are not working with external photon lines. Recall the path integral in a background gauge field;
\begin{equation} \label{bgfpath}
Z = \int D \overline{\psi} D \psi \exp{
i \int d^d x \overline{\psi} (i\slashed{D} - m) \psi.}
\end{equation}
There is no integration over $DA$ in the path integral - it is something in the background. So we just have a quadratic $Q$ in the Lagrangian of the form $i\slashed{D} - m$; fermionic integrals of the form \ref{bgfpath} just contribute a determinant. So it becomes
\begin{equation} \label{ferdet}
Z = \det (i \slashed{D} - m).
\end{equation}
Now, we know at the end of the day, that we have to normalize \ref{ferdet} via the free $Z$ (without a background $A$ interaction). So we write
\[
Z = \det (i \slashed{D} - m) = \det (i\slashed{\partial} - m - e \slashed{A}) = \det(i \slashed{\partial} - m) \det (1 - \frac{i}{i \slashed{\partial} - m} (-ie\slashed{A})).
\]
So after normalizing $Z$ it is apparent that we are interested in
\begin{equation} \label{ferdet1}
\det \left(
1 - \frac{i}{i \slashed{\partial} - m} (-ie\slashed{A})
\right)
\end{equation}
Now we've seen things like $\det (1 - X)$ in the past. For this reason we write
\[
\det B = \exp (\tr \log B),
\]
so that we may write
\[
\log( 1 - X )= -X
\]
Then the determinant \ref{ferdet1} as
\begin{equation} \label{ferdet2}
\det \left(
1 - \frac{i}{i \slashed{\partial} - m} (-ie\slashed{A}) \right) = 
\exp  \left[
\sum_{n=1}^\infty -\frac{1}{n} \tr \left[
\left(
\frac{i}{i\slashed{\partial} - m} (-ie\slashed{A})
\right)^n
\right]
\right].
\end{equation}
This is an exact result. However, we are interested only in a few terms. As we will see, the exact expression \ref{bgfpath} will soon obtain a regulator which wipes out most of the terms in \ref{ferdet2}. 

I will skip some steps here, but we find that the diagram

(page 305 in P{\&}S, 9.80)

is given by
\[
-\frac{1}{n} \text{Tr} \, \left[
\left(
\frac{i}{i\slashed{\partial} - m} (-ie\slashed{A})
\right)^n
\right].
\]
For now, I won't insert the Pauli-Villars regulators; I will pretend I don't need them. We are interested in the vacuum polarization diagram (pns305,9.80 with 2 external photon lines), which is given by
\begin{equation}
-\frac{1}{2} \text{Tr} \, \left[
\left(
\frac{i}{i\slashed{\partial} - m} (-ie\slashed{A})
\right)^2
\right].
\end{equation}
This may also be written less compactly as 
\[
= -\frac{1}{2} \int dx dy \tr \left[ (-ie\slashed{A} (x) ) S_F (y -x) (-ie\slashed{A} (y) S_F (x - y) \right],
\]
with $\tr$ the trace over spinor indices, and $S_F$ the usual fermion propagator
\[
S_F (x_1 - x_2) = \int \frac{d^d k}{(2\pi)^d} \frac{i e^{-ik (x_1 - x_2)}}{\slashed{k} - m + i\epsilon}.
\]
\end{document}