\documentclass[main.tex]{subfiles}

\begin{document}

maybe include some intro about the poincare group.

\chapter{Relativity in Quantum Mechanics}

Quantum field theory arises as a result of trying to construct relativistic quantum theories. Relativistic means invariance under Poincare transformations. Quantum means, at the level of states, that a Hilbert space is involved. Let's try to think about what this Hilbert space would look like. 

\section{Single Particles}
First, we will restrict ourselves to the single particle subspace. We take as our (nearly) complete set of commuting observables the four momenta $P^\mu$. We start with indices upper so that $p \to \Lambda p$, otherwise we'd get an inverse. We also consider the possibility of other degrees of freedom, which we label with $\sigma$. Therefore, we label our states
\[
\ket{p^0,p^1,p^2,p^3} = \ket{p,\sigma}.
\]
This state satisfies, then,
\[
P^\mu \ket{p} = p^\mu \ket{p,\sigma}.
\]
Now let's consider what should happen to these states under Lorentz transformations. Clearly, if I apply a transformation $\Lambda$ to my coordinates, I expect this state to change. Clearly we should have $p \to \Lambda p$. However, it could also mix the $\sigma$, and the way it does this could perhaps depend on both $p$ and $\Lambda$ independently. To account for this, we therefore have:
\[
\ket{p,\sigma} \to \sum_{\sigma'} C_{\sigma' \sigma} (\Lambda,p) \ket{\Lambda p,\sigma'}
\]
under a boost $\Lambda$. In order for the probabilities of outcomes to be invariant under Lorentz transformations, as well as Poincare transformations in general, transformations must be implemented by unitary matrices. That is, in order for things like $\bra{in}\ket{out}$ to be invariant, we must have
\[
\ket{in} \to U(\Lambda,a) \ket{in}, \quad \ket{out} \to U(\Lambda,a) \ket{out}
\]
for the same operator $U(\Lambda,a)$, with $a$ some translation parameter.  Therefore, a relativistic quantum theory enforces for single particle momentum eigenstates the following transformation rule:
\begin{equation} \label{eq:qmrelstate} 
\boxed{U(\Lambda) \ket{p,\sigma} = \sum_{\sigma'} C_{\sigma \sigma'} (\Lambda,p) \ket{\Lambda p, \sigma'}}
\end{equation}
Since the left hand side is implemented by group representations, the matrices $C(\Lambda,p)$ must also be a representation of the Lorentz group for each $p$. Let's classify their irreducible representations to get a better sense of what kind of relativistic quantum state spaces we have; the reducibles will break into a direct sum of these.

\subsection{The $C$ matrix}
Classification of these matrices amounts to classification of the coefficients. In order to do this, we stare at Eq. (\ref{eq:qmrelstate}). Let's take some momentum $k$. Consider a $p$ related to $k$ by $p = L(p) k$, with $L(p)$ some Lorentz transformation. Apply the transformation $U(\Lambda)$ on $k$. Then we get
\[
U(L(p))\ket{k,\sigma} = \sum_{\sigma'} C_{\sigma' \sigma} (L(p),k) \ket{p,\sigma'}.
\]
By linearity on the right hand side, this is an eigenstate of $P^\mu$ with eigenvalue $p^\mu$. 

\section{field transformation equations}
There are certain things where I'm not sure of a rigorous justification. I will indicated this with a [?]. However, I want at least some heuristic justification to myself of the equation
\begin{equation} \label{desiderata} 
U(\Lambda)^{\dagger} \phi(x) U(\lambda) = \phi(\Lambda^{-1} x).
\end{equation}
Where $U(\Lambda)$ is some unitary representation of the Lorentz group.

I can think of a pretty good reason. Let $\phi$ be a quantum field. This means we have an operator at each point $x$ in spacetime, $\phi(x)$. In the single particle quantum mechanics case, we would build general states out of position eigenstates $\ket{x}$. 


Now, if we are not creating scalar fields, we have to be a little more careful. We might want to create a vector valued excitation or something. In that case, we would, for example, need to consider how the field rotated when we apply some rotation. If I have fields $\phi_\mu$ that I want to transform like a four-vector, that means I want them to rotate and boost the way I expect as I rotate and boost. Let's forget the nuance of relativity for a second and just picture creating a unit vector field at a point $\vec{x} \neq 0$ pointing in the $x$ direction. The equations
\begin{equation} \label{whats}
U(R)^\dagger \phi_a (x) U(R) \ket{0} = \phi_a (R^{-1} x) 
\end{equation}
would not do the right thing - I'd just translate all the arrows around in a weird disgusting smushy vortical way that would only look right again after a $2\pi$ rotation. So we replace the right hand side of Equation \ref{whats} with an $R_{ab} \psi_b$ instead of a naked $\psi_a$. If I have fields $\phi_\mu$ that I want to transform like a four-vector, that means I want them to rotate and boost the way I expect as I rotate and boost.
\[
U(\Lambda)^\dagger \phi^\mu (x) U(\Lambda) = {\Lambda^\mu}_\nu \phi^\nu (x)
\]
At low energies, the $W^\pm$ and $Z$ bosons are described by these fields. \ref{co2}

Let's pause. You might be thinking to yourself,  ``a lot of this is something you might only guess if somebody gave you the answer.'' I will not put up a front. That is exactly what I'm doing here. I will supplement this with resources that explain this in better detail. [?] \footnote{I have not. Find resources that flesh this out in more detail. To be honest, I don't need much more to be convinced.} However, for now, I will proceed to wave my hands around, as a physicist is wont to do.

We may want to generalize this somewhat. We've seen scalar fields. We've seen four-vector fields. What other fields might we want to consider? The obvious next extension is tensor fields. You guessed it: the transformation equation is
\[
U(\Lambda)^\dagger T^{\mu \nu} (x) U(\Lambda) = {\Lambda^\mu}_\rho {\Lambda^\nu}_\sigma T^{\rho \sigma} (\Lambda^{-1}x).
\]
How much more general can this get? Let's consider that we have some fields $\phi_a$ and be dumb about where to put indices for a bit. Let us write $D(\Lambda)_{ab}$ for the matrix that will mix up the fields, which depends on what kind of transformation we do.\footnote{I do not consider a $D$ that depends on $x$ as well. It seems this would break Lorentz invariance in some way.} In formulae then, we expect something like
\begin{equation} \label{allfields}
\boxed{U(\Lambda)^\dagger \phi_a (x) U(\Lambda) = D(\Lambda)_{ab} \phi_b (x)}
\end{equation}
This is the most general type of field transformation we will consider. How general is it? Are there any constraints we can put on $D$? Well, stare at Equation \ref{allfields} for a bit. 

$U$ is already a group homomorphism. First, let's do nothing, $\Lambda = I$. Then $U(I) = I$. So
\[
\phi_a (x) = D(I)_{ab} \phi_b (x),
\]
so we need $D(I)_{ab} = \delta_{ab}$.

If we do two transformations, we have
\[
U(\Lambda_2)^\dagger U(\Lambda_1)^\dagger \phi_a (x) U(\Lambda_1) U(\Lambda_2) = D(\Lambda_1 \Lambda_2) \phi_b (\Lambda_2^{-1} \Lambda_1^{-1}x)
\]
The left hand side has a term in the middle that we already understand:
\[
U(\Lambda_2)^\dagger D(\Lambda_1)_{ac} \phi_c(\Lambda_1^{-1} x) U(\Lambda_2) = D(\Lambda_1 \Lambda_2)_{ab} \phi_b (\Lambda_2^{-1} \Lambda_1^{-1} x).
\]
Let's work with the left hand side for a second. $D_ac$ is just a number. So the $U(\Lambda_2)^\dagger$ on the far left can ignore it. Using the same trick as above we ged
\[
D(\Lambda_1)_{ac} D(\Lambda_2)_{cb} \phi_b (\Lambda_2^{-1} \Lambda_1^{-1} x) = D(\Lambda_1 \Lambda_2)_{ab} \phi_b (\Lambda_2^{-1} \Lambda_1^{-1} x).
\]
But the left hand side is just matrix multiplication in indices. So we must have
\[
D (\Lambda_1 \Lambda_2) = D(\Lambda_1) D(\Lambda_2).
\]
We may similarly check that $D(\Lambda^{-1}) = D(\Lambda)^{-1}$. So $D$ must also be a representation of the Lorentz group. Does it have to be unitary? Of course not - that constraint never came up in my discussion here. For one, there are no finite dimensional irreducible\footnote{Why do we care about irrep?} representations of the Lorentz group. Second, and more importantly, we won't be able to describe particles we see in Nature if we even tried to impose such a constraint.

\end{document}
