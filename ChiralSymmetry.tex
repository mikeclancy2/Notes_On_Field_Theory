\documentclass[main.tex]{subfiles}
\begin{document}
\chapter{Chiral Symmetry}
I'm working through Kaplan's notes and need to rewrite them in my own words. I will probably reformat it later.

In order to understand where chiral symmetry comes from, a review of the Lorentz group is in order.

\section{Spinor Representations of the Lorentz group}
Any time someone says ``spinor representations of the Lorentz group'', that should signal to you that it is a representation of the $SL(2,\cm)$. In general, when someone mentions spinor representations of a group, you should be thinking about the representations of its universal covering space.
\end{document}