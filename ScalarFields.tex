\documentclass{book}
\usepackage[utf8]{inputenc}
\usepackage[utf8]{inputenc}
\usepackage{geometry}
 \geometry{
 a4paper,
 total={170mm,247mm},
 left=20mm,
 top=20mm,
 }
\usepackage[utf8]{inputenc}
\usepackage{braket}
\usepackage{amsmath}
\usepackage{amssymb}
\usepackage{amsthm}
\usepackage{mathtools}
\usepackage{amsfonts}
\usepackage{physics}
\usepackage{hyperref}
\usepackage{tikz-cd}
\usepackage{enumerate}
\usepackage{tikz-feynman}
\tikzfeynmanset{compat=1.0.0}
\usepackage{graphicx}
\usepackage{slashed}
\usepackage{xfrac}
%\usepackage{xcolor}
%\pagecolor[rgb]{0,0,0}
%\color[rgb]{0,0.7,0.7}
\newcommand{\cm}{\mathbb{C}}
\newcommand{\re}{\mathbb{R}}
\newcommand{\mb}{\mathbf}
\newcommand{\proj}[1]{\ensuremath{{\ket{#1}\bra{#1}}}}
\newtheorem{defn}{Definition}
\newtheorem{prop}{Proposition}
\newtheorem{rmk}{Remark}
\newtheorem{conj}{Conjecture}
\newtheorem{lemma}{Lemma}
\newtheorem{cor}{Corollary}
\newenvironment{claim}[1]{\par\noindent\textit{Claim.}\space#1}{}
\newenvironment{claimproof}[1]{\par\noindent\textit{Subproof.}\space#1}{\hfill $\square$}
\renewcommand{\qedsymbol}{\ensuremath{\blacksquare}}

\begin{document}
\chapter{Scalar fields}
Most of the essential properties of quantum field theory, as well as its main technical challenges, can be understood in the context of scalar fields, where calculation is much simpler than in spinor or vector fields. Although there is only one scalar field in the standard model (the Higgs field), there is at least one scalar field in the standard model, so this exercise is more than just pedagogical. Furthermore, at certain low energy limits we can treat various fields as being scalar, i.e. the electromagnetic field (citation?). 

\section{The free real scalar field}
Let's begin with a real scalar field in $d = 4$ dimensions. There are motivations for the form of the following Lagrangian, but they aren't relevant here. We start with the Lagrangian
\[
\mathcal{L}_0 = - \frac{1}{2} \partial^\mu \phi \partial_\mu \phi - \frac{1}{2} m^2 \phi^2
\]
As a note, looking at the kinetic or mass term, we must have $[\phi] = \frac{1}{2}(d - 2) = 1$. The $0$ superscript on $\mathcal{L}$ denotes the lack of interactions.

The generating functional is 
\[
Z[J] = \frac{1}{Z[0]} \int D\phi \, \exp{i \int d^4x \, \left( \mathcal{L}_0 + J \phi \right)} 
\]
As review, by doing this as a Gaussian integral, this can be written
\[
Z[J] = \exp{
-\frac{1}{2} \int d^4 x d^4 y J(x) \Delta_F (x - y) J(y)
}
\]
where
\[
\Delta_F (x-y) = \int \frac{d^4 k}{(2\pi)^4} e^{-ik(x-y)} \frac{i}{k^2 - \mu^2 + i\epsilon}
\]
This is where we left off in the functional integrals section. Now, you don't have to do this integral exactly to know it's going to be a problem. $d^4 k$ gives us something like $k^4$, and the denominator only gives us $1/k^2$. You might think that the phase from the plane wave could help us cancel out, but this is going to go in the LSZ formula: those aren't phases, they're half of a delta function.
\section{To Do List}
\begin{itemize}
\item add an intro section about path integrals
\end{itemize}
\end{document}