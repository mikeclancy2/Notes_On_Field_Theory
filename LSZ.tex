\documentclass{book}
\usepackage[utf8]{inputenc}
\usepackage[utf8]{inputenc}
\usepackage{geometry}
 \geometry{
 a4paper,
 total={170mm,247mm},
 left=20mm,
 top=20mm,
 }
\usepackage[utf8]{inputenc}
\usepackage{braket}
\usepackage{amsmath}
\usepackage{amssymb}
\usepackage{amsthm}
\usepackage{mathtools}
\usepackage{amsfonts}
\usepackage{physics}
\usepackage{hyperref}
\usepackage{tikz-cd}
\usepackage{enumerate}
\usepackage{tikz-feynman}
\tikzfeynmanset{compat=1.0.0}
\usepackage{graphicx}
\usepackage{slashed}
\usepackage{xfrac}
%\usepackage{xcolor}
%\pagecolor[rgb]{0,0,0}
%\color[rgb]{0,0.7,0.7}
\newcommand{\cm}{\mathbb{C}}
\newcommand{\re}{\mathbb{R}}
\newcommand{\mb}{\mathbf}
\newcommand{\proj}[1]{\ensuremath{{\ket{#1}\bra{#1}}}}
\newtheorem{defn}{Definition}
\newtheorem{thm}{Theorem}
\newtheorem{prop}{Proposition}
\newtheorem{rmk}{Remark}
\newtheorem{conj}{Conjecture}
\newtheorem{lemma}{Lemma}
\newtheorem{cor}{Corollary}
\newenvironment{claim}[1]{\par\noindent\textit{Claim.}\space#1}{}
\newenvironment{claimproof}[1]{\par\noindent\textit{Subproof.}\space#1}{\hfill $\square$}
\renewcommand{\qedsymbol}{\ensuremath{\blacksquare}}

\begin{document}
\chapter{LSZ Reduction}
In this chapter we describe the procedure known as the LSZ Reduction Formula, due to Lehmann, Symanzik, and Zimmermann. It is hard to describe just how indispensable this formula is for computing physical quantities. At the end of the day, the ways in which quantum field theory is tested is in scattering processes. I throw particles together, and I predict what comes out in terms of momenta. What we care about then, is the overlap of the initial and final state of a system. This is given by (S matrix equation, Dyson's equation maybe? think about how to do this)

\begin{itemize}
\item spin 0 (recreate srednicki)
\item spin 1/2 (recreate srednicki)
\item general case (recreate weinberg)
\end{itemize}
\end{document}