\documentclass[main.tex]{subfiles}

\begin{document}
\chapter*{Introduction}
Welcome to my notes on field theory.

The first purpose of these notes is to serve as a resource for myself to thoroughly imprint quantum field theory into my brain by writing. The (chronologically) second, but equally important, purpose is to serve as a reference for myself when I need to review field theory. The third purpose is as a supplementary resource for a physics graduate student learning quantum field theory for the first or second time. I attempt to cover all of the main topics in broad strokes, but lack the depth and granularity of some of the actual texts on field theory. The graduate student would find chapters like the Gaussian Integral or Functional Calculus chapters most useful in their study of a standard text, as topics like functional calculus are skimmed over typically. In short, if it feels like your text is missing something important, I hope you will be able to find it here.

I tried to organize the chapters in a way that made the most sense to me. There is an inherent trickiness in communicating this subject linearly to someone who doesn't already know all of math and physics. So I sometimes claim things that I can only partially justify, and later fill in the details. I have tried my best to arrange this in a linear fashion. These notes are divided into four parts. The first part contains chapters which communicate the big ideas. This involves things like Lagrangians, Lorentz invariance, path integrals, and S-matrices. The second part puts together some of the math needed to actually construct these theories, and introduces  some field theory along the way. The third section will merge the previous two, applying the techniques of the second section to get a handle on the ideas of the first section. This is where we do (the big stuff). The fourth section will address some advanced topics. This section is the largest, as we can finally do quantum field theory after the first three sections.

These notes were assembled largely through the reading of Coleman's notes (2019), Peskin \& Schroder, and Srednicki. Occasionally I will use other references.
\end{document}