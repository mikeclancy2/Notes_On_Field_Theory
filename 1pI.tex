\documentclass{book}
\usepackage[utf8]{inputenc}
\usepackage[utf8]{inputenc}
\usepackage{geometry}
 \geometry{
 a4paper,
 total={170mm,247mm},
 left=20mm,
 top=20mm,
 }
\usepackage[utf8]{inputenc}
\usepackage{braket}
\usepackage{amsmath}
\usepackage{amssymb}
\usepackage{amsthm}
\usepackage{mathtools}
\usepackage{amsfonts}
\usepackage{physics}
\usepackage{hyperref}
\usepackage{tikz-cd}
\usepackage{enumerate}
\usepackage{tikz-feynman}
\tikzfeynmanset{compat=1.0.0}
\usepackage{graphicx}
\usepackage{slashed}
\usepackage{xfrac}
%\usepackage{xcolor}
%\pagecolor[rgb]{0,0,0}
%\color[rgb]{0,0.7,0.7}
\newcommand{\cm}{\mathbb{C}}
\newcommand{\re}{\mathbb{R}}
\newcommand{\mb}{\mathbf}
\newcommand{\proj}[1]{\ensuremath{{\ket{#1}\bra{#1}}}}
\newtheorem{defn}{Definition}
\newtheorem{prop}{Proposition}
\newtheorem{rmk}{Remark}
\newtheorem{conj}{Conjecture}
\newtheorem{lemma}{Lemma}
\newtheorem{cor}{Corollary}
\newenvironment{claim}[1]{\par\noindent\textit{Claim.}\space#1}{}
\newenvironment{claimproof}[1]{\par\noindent\textit{Subproof.}\space#1}{\hfill $\square$}
\renewcommand{\qedsymbol}{\ensuremath{\blacksquare}}

\begin{document}

\chapter{Taming Feynman Diagrams: 1PI, Loop Corrections}
We're going to do something important. We're going to compute the exact propagator 
\[
\frac{1}{i}\mathbf{\Delta}(x-y) = \bra{0} T \phi(x) \phi(y) \ket{0}
\]
We know, in principle, that we should compute this order by order in $g$ by using Feynman diagrams. It is also easier in general to write it in momentum space. This will matter in a little but. In any case we have an expansion
\[
\frac{1}{i}\mathbf{\Delta}(k^2) = \sum_n g^n \times \left( \sum \text{diagrams with $g^n$ factor} \right)
\]
so we have a double sum. For small $g$, it makes sense to calculate everything up to some order $g^n$, and then stop. To do this, we need to sum up each diagram of order $n$. For small $n$, it's clear how many diagrams contribute. However, it's not clear that there's a nice closed form expression for the exact propagator by using diagrams.

The first realization we should make is this: infinite sums that we don't compute by approximating with the first few terms are familiar. If we see something like
\[
\sum_n r^n,
\]
for $0 < r < 1$, we know that this sum is just
\[
\frac{1}{1-r}.
\]
In fact, this is more than an instructive example: this is exactly what we will do. We are going to compute diagrams by doing a geometric series. We will get to that shortly. I will explain first how we are led there. \newpage

\section{1PI diagrams}
Let's say I draw a diagram. Here's one:

(one $\phi^3$ loop diagram) (A)

Very cool. This diagram is easy. Let's do an ugly one:

(hammock $\phi^3$ diagram: line, triangle, square, triangle, line) (B)

Gross. I've got to keep track of a whole bunch of terms. What about this one?

(external prop, $\phi^3$ loop, internal prop, hammock, external prop) (A+B)

This actually isn't bad if we've already done the previous two: it's just
\[
\text{prop times loop times prop times hammock times prop}
\]
In other words, any time a diagram consists of two ugly pieces connected by a single internal line, the diagram factors as above. We will organize diagrams in this way, but we have to do so systematically: so even if a subdiagram is ugly, I want to make sure I check if it can also be seen as two ugly diagrams connected by a single internal line. This motivates the following definition:

\begin{defn}
A diagram is called \textbf{one-particle irreducible}, or just \textbf{1PI}, if cutting a single internal line doesn't separate it into two disconnected diagrams.
\end{defn}
So (A), (B) are 1PI, while (A+B) is clearly not 1PI. 

By the definition of 1PI, we have that any diagram is a bunch of 1PI's connected in a chain by internal lines. Picture these as little knots. If we want to sum over all diagrams, we are motivated to define the following quantity:
\[
i\Pi(k^2) = \sum(\text{all 1PI diagrams with incoming \& outgoing momentum $k$})
\]
We refer to $i\Pi^2$ as the \textit{self-energy}, because it consists of all internal processes of a single particle coming in and going out, without existing as a single particle at any point inbetween.

So, if we want to add up all possible diagrams, we organize terms by how many 1PIs there are connected by links:

(Srednicki picture)

Therefore we may write
\begin{align}
&\frac{1}{i} \mathbf{\Delta} (k^2) = &\\ & \frac{1}{i}\Delta(k^2) +& \\ 
&\frac{1}{i}\Delta(k^2) (i \Pi(k^2)) \frac{1}{i}\Delta(k^2) + & \\ 
&\frac{1}{i}\Delta(k^2) (i\Pi(k^2)) \frac{1}{i}\Delta(k^2)  (i \Pi(k^2)) \frac{1}{i}\Delta(k^2) + & \\
&\cdots &
\end{align}
Now comes a really beautiful idea. Nothing here is an operator or anything stupid like that. It's all numbers. So this is really just a geometric series! The exact propagator  given by
\[
\mathbf{\Delta} = \Delta \frac{1}{1-\Delta\Pi}
\]
But $\Delta = 1/x$ for $x = k^2 + m^2 - i\epsilon$. So we can write this as 
\[
\mathbf{\Delta} = \frac{1}{x - \Pi}
\]
Rewriting this in full notation we have:
\begin{equation} \label{eq:wowgeo}
\mathbf{\Delta} (k^2) = \frac{1}{k^2 + m^2 - i\epsilon - \Pi(k^2)}
\end{equation}
This is kind of incredible. In order to compute the exact propagator, we can just compute the self-energy order by order.
\end{document}